\documentclass[aspectratio=169,11pt,hyperref={colorlinks=true}]{beamer}
\usepackage[utf8]{inputenc}
\usepackage[T1]{fontenc}
\usepackage{fontspec}
\usepackage[absolute,overlay]{textpos}
\usepackage{listingsutf8}
\usepackage{listings-golang}
\usepackage{tikz}
\usepackage{color}


\title{Developing a CD pipeline with Knative}
\date[DevOps Meetup]{March 14th 2019, Singapore, Singapore}
\author[Andrea]{
  Andrea Frittoli \\
  Developer Advocate \\
  andrea.frittoli@uk.ibm.com \\
  @blackchip76
}
\institute[DevOpsMeetupSingapore]{
  DevOps Meetup Singapore
}

\usetheme{ibmcloud}

% Code style
\setlststyle

\lstdefinelanguage{koyaml}{
  keywords={github, com, afrittoli, examples, ms, go, helloworld},
  sensitive=false,
  comment=[l]{\#},
  morestring=[b]',
  morestring=[b]"
}

% Automatic section frame
\AtBeginSection{\frame{\sectionpage}}

\begin{document}

\begin{frame}[noframenumbering]
\titlepage{}
\end{frame}

% The main points of the talk are:
% - give an introduction to Tekton Pipelines
% - get in some more details with an example
% - talk about Kaniko and source to image with tricky bits
% - reusing tasks, dev vs CI vs CD
% - point of integration with serving and eventing
% - can I use Tekton? Where? security concerns

% The slide/section order does not match the narrative yet

\section{A Bit of History}

\begin{lblackrwhiteframe}
  \frametitle{Knative}
  \large
  \begin{beamercolorbox}[wd=0.3\paperwidth]{text}
    \begin{itemize}
      \item Beginning of 2018...
      \item Knative:
      \begin{itemize}
        \item Build
        \item Eventing
        \item Serving
      \end{itemize}
    \end{itemize}
    \begin{itemize}
      \item Contributors:
      \begin{itemize}
        \item Google
        \item Pivotal
        \item IBM
        \item RedHat
        \item ...and others
      \end{itemize}
    \end{itemize}
  \end{beamercolorbox}%
  \begin{textblock*}{0.5\paperwidth}(0.5\paperwidth,0.2\paperheight)
    \centering
    \includegraphics[width=0.45\paperwidth]{img/knative-audience.png}
  \end{textblock*}
\end{grayframe}

\begin{blackframe}
  \frametitle{\textasciitilde Sept 2018: Knative Pipelines}
  \begin{textblock*}{\paperwidth}(0cm,0.2\paperheight)
    \includegraphics[width=\paperwidth]{img/pipeline_cc0.jpg}
    % https://mediad.publicbroadcasting.net/p/shared/npr/styles/placed_wide/nprshared/201804/605180710.jpg
  \end{textblock*}
  \begin{textblock*}{0.2\paperwidth}(0.83\paperwidth,0.93\paperheight)
    \includegraphics[width=0.03\paperwidth]{img/cc.png}
    \includegraphics[width=0.03\paperwidth]{img/zero.png}
  \end{textblock*}
\end{blackframe}

\begin{lblackrwhiteframe}
  \frametitle{\textasciitilde Feb 2019: Tekton Pipelines}
  \large
  \begin{beamercolorbox}[wd=0.3\paperwidth]{text}
    \begin{itemize}
      \item Focus on CI/CD
      \item Deploy ``anywhere``
      \item Compatible with Knative Build
    \end{itemize}
    \begin{itemize}
      \item {\em tektoncd/pipeline}
      \begin{itemize}
        \item Logo TBD
        \item Governance WIP
        \item Alpha APIs
        \item Roadmap WIP
      \end{itemize}
    \end{itemize}
  \end{beamercolorbox}%
  \begin{textblock*}{0.5\paperwidth}(0.5\paperwidth,0.35\paperheight)
    \centering
    \includegraphics[width=0.35\paperwidth]{img/tekton_not_knative.png}
  \end{textblock*}
\end{lblackrwhiteframe}

\begin{grayframe}
  \frametitle{Community}
  \begin{itemize}
    \item {\em Valid for Knative. Tekton TBD.}
    \item Steering Commitee (SC)
    \item Technical Oversight Commitee (TOC)
    \item Various Contribution profiles
    \item Design, issues: on GitHub
    \item Communication:
    \begin{itemize}
      \item Weekly video meetings, recorded, Build WG
      \item Asynch: Knative Users / Developers ML
      \item Sync: slack.knative.dev
    \end{itemize}
  \end{itemize}
\end{grayframe}

\section{Tekton Pipelines}

\begin{grayframe}
  \frametitle{Intro}
  % Cloud Native Pipelines, runs on Kubernetes
  % Diagram with step, task, pipeline, input, output, params
  % -> Use colour to identify runtime elements
  % -> Use colour to identify CRDs
\end{grayframe}

\begin{grayframe}
  \frametitle{The Health Application}
  % architecture graph
  % screenshot
\end{grayframe}

\begin{grayframe}
  \frametitle{Inputs, Outputs \& DAG}
  % Graph of a simple health pipeline
\end{grayframe}

\section{Under the Hood}

\begin{grayframe}
  \frametitle{Custom Resources}
  % Webhook validation
  % Controller reconciles resources -> runs tasks and pipelines
  % Defined as YAML
  % -> From file
  % -> From an app, using go API
\end{grayframe}

\begin{grayframe}
  \frametitle{Pods, Entrypoints \& Volumes}
  % Task -> Pod (one node)
  % Step -> Container (custom images)
  % Entrypoint to order containers, rewrite entrypoint
  % One pod, scheduled once, same node
  % Empty dir can be shared
  % Pipeline -> Multiple Pods. PVC vs Object storage
  % maybe another diagram
\end{grayframe}

\section{Source to Image to Deploy}

\begin{grayframe}
  \frametitle{IBM Cloud}
  % Private registry
  % Service accounts
  % Experimental add-on, install knative
\end{grayframe}

\begin{grayframe}
  \frametitle{CD Pipeline as code}
  % Everything is defined via YAML
  % Declarative: pipeline as code, reconciled
  % Git PipelineResource
  % Resources and parameters
  % -> Environment, run specific, secrets
\end{grayframe}

\begin{grayframe}
  \frametitle{Using Kaniko}
  % Caching layers
  % Base images cache
  % Output image
  % Optimize the docker image
  % Reproducible builds
  % Alternatives for build
  % Image Pipeline Resource
  % Kaniko Logo
\end{grayframe}

\begin{grayframe}
  \frametitle{Cluster resource and secrets}
  % Deploy to cluster
  % -> cluster resource
  % Security considerations
  % Targeting a namespace
  % Using service accounts
\end{grayframe}

\section{Tekton and Knative}

\begin{grayframe}
  \frametitle{Pipelines and Knative Build}
  % *Run can be used as build with serving via duck typing
  % Build probably won't be developed much further
  % ...but you never know
  % Do something more complex with pipelines
\end{grayframe}

\begin{grayframe}
  \frametitle{CI for OpenStack Health}
  % There's already CI, with Zuul, check it out
  % Pipeline as code... again
  % GitOps Style (check gitops)
  % Diagram of CI pipeline for OH
  % Reusing tasks, best practices
\end{grayframe}

\begin{grayframe}
  \frametitle{CI with Tekton Pipelines}
  % It's not a CI engine...
  % ...but it works well with one
  % Dogfooding
  % What about secrets?
  % Existing integrations
  % - jenkinsX, prow
  % - gitlab? pivotal? to be checked
  % - higher level abstractions
\end{grayframe}

\section{Asynchronous Pipelines}

\begin{grayframe}
  \frametitle{Triggering and Knative Eventing}
  % Manual trigger
  % Using tasks and pipelines in Kservices
  % Native triggers: TBD
  % Async pipelines
  % GitHub events
  % -> push/pull request (CI)
  % -> comment (CI / CD)
  % -> release (CD)
\end{grayframe}

\begin{grayframe}
  \frametitle{Tekton and Development}
  % Maybe
  % Local development with minikube
  % Skaffold, ko, gulp
  % Pros and cons
  % This slide breaks the narrative here
\end{grayframe}

\section{Conclusions}

\begin{grayframe}
  \frametitle{Shall I use Tekton Pipelines?}
  % -Alpha APIs
  % +Cloud Native
  % +Relatively small footprint
  % +Momentum
  % +Reusable bits, library of Tasks
  % ...so it depends
\end{grayframe}

\begin{grayframe}
  \frametitle{Roadmap}
\end{grayframe}

\begin{grayframe}
  \frametitle{References}
  % https://github.com/tektoncd/pipeline
  % https://github.com/tektoncd/pipeline/blob/master/api_compatibility_policy.md
  % https://github.com/tektoncd/pipeline/blob/master/roadmap-2019.md
  % https://github.com/knative/docs/tree/master/community
  % links to blog posts
  % Link to kaniko
\end{grayframe}

\section{Q\&A}

\end{document}
